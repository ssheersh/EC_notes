\documentclass[12pt]{article}
\usepackage{amsmath}
\usepackage{amssymb}

\author{Samyak Sheersh}
\date{\today}
\title{Analog Electronics Notes}


\begin{document}
\maketitle
\tableofcontents

\section{Some basic concepts}
To manipulate currents externally, we generally need to create a gradient, which generates an electric field which can be controlled externally

Band gap in Si $=1.12 eV$\\
$\Rightarrow \text{For Silicon to function as a semiconductor, } \frac{1}{2}k_{B} T\geq 1.12 eV$\\

where $k_B$ is the Boltzmann constant.

In a simple p-n junction, the p-n junction is a metallurgical junction, and at thermal equilibrium, it forms a depletion layer, which results in a built-in potential $V_{bi}$, due to a distribution of charge. 

If the p-side is doped with $N_a$ acceptor atoms($cm^{-3}$) and the n-side is doped with $N_d$ donor atoms($cm^{-3}$), then:
\begin{equation}
    V_{bi}=\frac{k_B T}{q}\ln(\frac{N_a N_d}{n_{i}^{2}})
\end{equation}
which at $T\approx 300 K$:
\begin{itemize}
    \item $V_T$ called the thermal voltage $q=e=1.6*10^{-19}$, $$V_T=\frac{k_B T}{q}\approx 26mV$$. 
    \item $n_i=1.35*10^{10}\approx 10^{10} cm^{-3}$
\end{itemize}
\end{document}

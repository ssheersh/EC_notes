\documentclass[12pt]{article}
\usepackage{header}
\author{Samyak Sheersh}
\date{\today}
\title{Analog Electronics Notes}


\begin{document}
\maketitle
\tableofcontents

\section{Some basic concepts}
To manipulate currents externally, we generally need to create a gradient, which generates an electric field which can be controlled externally

Band gap in Si $=1.12 eV$\\
$\Rightarrow \text{For Silicon to function as a semiconductor, } \frac{1}{2}k_{B} T\geq 1.12 eV$\\

where $k_B$ is the Boltzmann constant.

In a simple p-n junction, the p-n junction is a metallurgical junction, and at thermal equilibrium, it forms a depletion layer, which results in a built-in potential $V_{bi}$, due to a distribution of charge. 

If the p-side is doped with $N_a$ acceptor atoms($cm^{-3}$) and the n-side is doped with $N_d$ donor atoms($cm^{-3}$), then:
\begin{equation}
    V_{bi}=\frac{k_B T}{q}\ln(\frac{N_a N_d}{n_{i}^{2}})
\end{equation}
which at $T\approx 300 K$:
\begin{itemize}
    \item $V_T$ called the thermal voltage $q=e=1.6*10^{-19}$, $$V_T=\frac{k_B T}{q}\approx 26mV$$. 
    \item $n_i=1.35*10^{10}\approx 10^{10} cm^{-3}$
\end{itemize}

Note that this voltage difference can not be used to extract energy if there's no temperature difference. (The Second Law of Thermodynamics)

\subsection{Law of Mass Action}
In equilibrium:
\begin{equation}
    n_i^2=n*p
\end{equation}
where n and p are the electron and hole concentration of a doped semiconductor

Generally, you dope it with either:
\begin{itemize}
    \item donor atoms, which causes it to become an n-type semiconductor

If it's doped with donor atoms, with concentration $N_d>>n_i$, then $n\approx N_d$ and $p\approx \frac{n_i^2}{N_d}$

    \item acceptor atoms, which causes it to become a p-type semiconductor

        If it's doped with donor atoms, with concentration $N_a>>n_i$, then $p\approx N_a$ and $n\approx \frac{n_i^2}{N_a}$
\end{itemize}

\subsection{Capacitance across the depletion layer}
Due to the build-up of charge across the depletion layer in reverse or zero bias (immobile as it may be), we can model this behaviour as a capactiance:
\begin{align}
    C_{t}=\frac{\epsilon A}{W_{dep}}\\
    W_{dep}=\sqrt{\frac{2\epsilon}{q}(V_{bi}+V_{R})\big(\frac{1}{N_a}+\frac{1}{N_d}\big)}
\end{align}
where $\epsilon$ is Silicon's permittivity, $W_{dep}$ is the width of the depletion layer and $V_R$ is the external voltage applied in reverse bias


\section{BJT and Biasing}


\end{document}
